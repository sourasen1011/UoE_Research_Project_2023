% ****** Start of file apssamp.tex ******
%
%   This file is part of the APS files in the REVTeX 4.2 distribution.
%   Version 4.2a of REVTeX, December 2014
%
%   Copyright (c) 2014 The American Physical Society.
%
%   See the REVTeX 4 README file for restrictions and more information.
%
% TeX'ing this file requires that you have AMS-LaTeX 2.0 installed
% as well as the rest of the prerequisites for REVTeX 4.2
%
% See the REVTeX 4 README file
% It also requires running BibTeX. The commands are as follows:
%
%  1)  latex apssamp.tex
%  2)  bibtex apssamp
%  3)  latex apssamp.tex
%  4)  latex apssamp.tex
%
\documentclass[%
 reprint,
 amsmath,amssymb,
 aps,
 nofootinbib,
]{revtex4-2}

\usepackage{graphicx}% Include figure files
\usepackage{dcolumn}% Align table columns on decimal point
\usepackage{bm}% bold math
\usepackage{amsthm}%theorems and stuff
\usepackage[ruled,lined]{algorithm2e}
\usepackage{svg}
\usepackage{float}
\usepackage[hidelinks]{hyperref}
\usepackage{color}
\usepackage{enumitem}
%Import the natbib package and sets a bibliography  and citation styles
\usepackage{etoolbox}
\usepackage{natbib}
% make the references numbered
\usepackage{hyperref,url}
\newcounter{bibcount}
\makeatletter
\patchcmd{\@lbibitem}{\item[}{\item[\hfil\stepcounter{bibcount}{\thebibcount.}}{}{}
\setlength{\bibhang}{2\parindent}
\renewcommand\NAT@bibsetup%
   [1]{\setlength{\leftmargin}{\bibhang}\setlength{\itemindent}{-\parindent}%
       \setlength{\itemsep}{\bibsep}\setlength{\parsep}{\z@}}
\makeatother
\bibliographystyle{agsm}
\setcitestyle{authoryear,open={(},close={)}} %Citation-related commands
\usepackage{pgfgantt} % Gantt chart
\usepackage{rotating}

\hypersetup{
    colorlinks=false,
    linkcolor=blue,
    filecolor=magenta,      
    urlcolor=cyan,
    pdftitle={My Proposal},
    pdfpagemode=FullScreen,
    }
\setlength\parindent{0pt} % zero indent
\theoremstyle{definition}
\newtheorem{definition}{Definition}[section]

%\usepackage[showframe,%Uncomment any one of the following lines to test 
%%scale=0.7, marginratio={1:1, 2:3}, ignoreall,% default settings
%%text={7in,10in},centering,
%%margin=1.5in,
%%total={6.5in,8.75in}, top=1.2in, left=0.9in, includefoot,
%%height=10in,a5paper,hmargin={3cm,0.8in},
%]{geometry}


% Keywords command
\providecommand{\keywords}[1]
{
  \small	
  \textbf{\textit{Keywords---}} #1
}

\begin{document}

\preprint{APS/123-QED}

\title{M.Sc. Project Proposal}% Force line breaks with \\
%\thanks{A footnote to the article title}%
\author{Souradeep Sen \\
	 \small Department of Computer Science, \\ 
	 \small University of Exeter
	}

\date{April, 2023}% It is always \today, today,
             %  but any date may be explicitly specified

\begin{abstract}
The study aims to compare the performance of a Hybrid Deep Learning (HDL) architecture for predicting mortality and hospitalization in heart failure (HF) patients with frailty, against traditional survival analysis techniques. HDL uses unsupervised and supervised layers to estimate prediction probabilities based on low dimensional and contextual/historic features from clinical data. By leveraging longitudinal patient data made available in the form of Electronic Health Records (EHR), a new perspective is sought on the performance of machine learning risk prediction models compared to conventional survival analysis in HF patients.
\end{abstract}

\maketitle

\section{\label{intro}Introduction}
Heart failure (HF) is a complex clinical syndrome that affects millions of individuals worldwide, particularly those with frailty, a common condition in older adults. Predictive models using machine learning (ML) based approaches have shown promise in identifying patients at high risk of adverse outcomes. In this project, we aim to develop and validate predictive models to identify HF patients with frailty who are at increased risk of mortality and hospitalization using EHR data. 

\section{\label{rescon}Research Context}
Traditional survival analysis has been used extensively to predict mortality in this patient population. Deep learning methods have also been employed in \citep{e2edlgjoreski}, \citep{nirschl2018deep}, \citep{10.1001/jamanetworkopen.2019.6972}, \citep{asolares2020} and \citep{lorenzoni_2019}. However, limited work has been done in predicting mortality and hospitalization in HF patients with frailty, especially using electronic health records (EHR) data.
\subsubsection{\label{frail}Who is a `frail' patient?}
The target pool of patients being those suffering from frailty, one first needs to understand what the term `frail' refers to. Although its meaning may be nebulous to some extent, the eFI (Electronic Frailty Index) is a good clinical approximation of the condition. The eFI is based on the cumulative deficit model of frailty \citep{NHS}, which proposes that frailty is the result of a build-up of health deficits or impairments. It is calculated using a set of clinical indicators, such as the presence of chronic conditions, cognitive impairment, and mobility problems, that are commonly documented in EHRs.
\subsubsection{\label{hf}Investigating Heart Failure}
HF can be influenced by lifestyle risk factors, such as diet, exercise, smoking, and alcohol consumption. Clinical factors such as diabetes, hypertension, BMI, cholesterol, and sugar consumption were also identified as potential risk factors for HF. Familial history may not be pertinent for all cases of HF, but may only be relevant for some cases - familial dilated cardiomyopathy (DCM) is a type of heart failure that can have a strong genetic component \citep{peters2020familial}.
\subsubsection{\label{pastwork}Literature Review}
Several past papers have addressed predictive modeling for heart failure patients. \citep{che2017} proposes a deep neural network model with learned medical feature embedding to address high dimensionality and temporality in electronic health record (EHR) data. They use a convolutional neural network to capture non-linear longitudinal evolution of EHRs and local temporal dependency for risk prediction, and embed medical features to account for high dimensionality. Experiments show promising results in predicting risks for congestive heart failure. \citep{suo2017personalized} focuses on personalized predictive modeling and aims to build specific models for individual patients using similar patient cohorts to capture their specific characteristics. According to this study, although CNNs have shown promise on measuring patient similarity, one disadvantage is that they could not utilize temporal and contextual information of EHRs. To measure patient similarity using EHRs, the authors proposed a time-fusion CNN framework. A vector representation was generated for each patient, which was then utilized for measuring patient similarity and personalized disease prediction. Dynamic updates to a CNN model are explored in \citep{brand2018real} as more data is gathered over time - this architecture lends itself well to real-time mortality risk prediction. Maintaining interpretability across deep learning models is explored in \citep{caicedo2019iseeu}. Many previous studies using machine learning for modeling the risk of HF in patients have focused on discretized outputs. This study aims to consider incidences as time-to-event to enable continuous probabilistic risk prediction for hospitalization and mortality, addressing a critical need in patient care. The use of EHR such as those available in CPRD, allows access to comprehensive longitudinal data, which captures the entire cycle of a patient's diagnosis and treatment. Limited work has been done on predictive modeling for HF patients with frailty, particularly using hybrid deep learning and Hidden Markov Models (HMMs). This project aims to bridge this gap in the literature.

\section{\label{aimsobj}Aim \& Objectives}

\subsection{Research Questions:}
\begin{enumerate}
\item To what extent can HDL predict HF as time-to-event?
\item How transparent/ interpretable can the HDL model be for prediction of HF?
\item What are the trade-offs between HMM and HDL with respect to interpretability and performance?
\end{enumerate}

\subsection{Objectives:}

This project aims to develop a scalable predictive model for frail patients with heart failure, combining deep neural networks with statistical machine learning. The results of this project could improve patient outcomes by identifying those at high risk and facilitating targeted interventions. The ML approach relies on hybrid deep learning, specifically an architecture that combines statistical learning in the lower unsupervised layers and convolutional neural networks on the top layer. A secondary objective is to compare the hybrid deep learning models with a more mathematically transparent model, Hidden Markov Models. If time permits, I will also assess the trade-off between model performance and interpretability to determine the most suitable model for this task.

\section{\label{datares}Data \& Resources}
CPRD (Clinical Research Practice Datalink) data is likely to be used for this project. From a collaboration with the medical school at the University of Exeter, I am being granted access to a set of data contaning EHR records of a vast number of patients, consisting of both HF patients and non-HF patients.

\subsection{\label{data}Data Sources}

\subsubsection{\label{cprd} Primary Data Source: CPRD Aurum}
The Clinical Practice Research Datalink (CPRD) is a large, longitudinal UK primary care database containing anonymized electronic health records. It includes information on patient demographics, diagnoses, symptoms, prescriptions, and referrals, among other clinical data. The database is widely used for observational research, including epidemiological studies, drug safety monitoring, and healthcare utilization analysis. CPRD is maintained by the Medicines and Healthcare products Regulatory Agency (MHRA) and is available to researchers and industry partners who meet certain criteria and obtain appropriate approvals.\\

The Clinical Practice Research Datalink (CPRD) Aurum is a database provided by EMIS Health. It includes data for approximately 7 million active patients, and covers consenting practices in England, with consenting practices from Northern Ireland available from 2019. CPRD Aurum is updated monthly, and it is representative of the broader English population in terms of geographical spread, age, and gender. However, it does not usually record data outside of primary care. It is important to be aware of this limitation. To bypass it, the CPRD Aurum database can be linked to datasets such as Hospital Episode Statistics (HES), Death Registration, Cancer data, Mental Health Services Dataset, and Small Area-Level Data.

\subsubsection{\label{mimiciii} Contingent Data Source: MIMIC-III}

The MIMIC-III (Medical Information Mart for Intensive Care III) dataset is a large, freely available critical care database that contains de-identified health data from over 60,000 patients admitted to the Beth Israel Deaconess Medical Center ICU between 2001 and 2012. The data is diverse, including demographic information, vital signs, laboratory tests, medications, and more. The dataset also includes information about procedures, diagnoses, and outcomes. The MIMIC-III dataset has been widely used for research in various fields, such as machine learning, clinical decision support, and epidemiology. The dataset has contributed to numerous studies, and many of its findings have led to improved patient care and clinical decision-making. The database has been made publicly available to researchers around the world, but access is restricted and requires a formal application process.

\subsection{\label{coding}Coding}

There are several possible tools of choice for the project. The ones I plan to use are Python for exploratory analysis, coding the base unsupervised layers and upper supervised layers as well as encoding contextual and historical data. This is because it has many useful libraries and frameworks at its disposal for numerical computing and neural networks, namely numpy, scipy, Pytorch and Tensorflow. For the more statistically inclined approach of Hidden Markov Models, I plan to use R as it has native functionalities for building Markov chains as well as pacakges like \texttt{brms} and \texttt{rstan} for Bayesian simulation methods like Markov Chain Monte Carlo.

\section{\label{methods}Methods \& Experiment Design}

\subsection{\label{build}Model Building}
To evaluate the performance of the hybrid deep learning models and Hidden Markov Models developed in this project, I will compare them to traditional survival analysis models such as Cox proportional hazards and Kaplan-Meier curves. These models are widely used in predicting outcomes in heart failure patients, providing a benchmark for ML-based models. I will assess the performance of the models based on metrics such as accuracy, sensitivity, and specificity. The comparison will allow us to determine whether the ML-based models outperform traditional models, providing insights into the potential of these models for predicting adverse outcomes in frail patients.
\subsubsection{\label{prim}Primary Aim: CNN}
The idea is to parse through time-series data of individual patient histories and identify signals that can discriminate between potential HF and non-HF patients. These signals can then be encoded from their high-dimensional space into a latent vector representation in lower dimensions - possible techniques for dimensionality reduction such as Principal Component Analysis (PCA) and Linear Discriminant Analysis (LDA) may be used for this purpose. These latent representations can then be fed into supervised layers of a Convolutional Neural Network. The output from the network should be an array depicting the persistency of a patient till an even of interest - in this case, time-to-hospitalization or time-to-death. Separate hold-out sets can be reserved for testing purposes. This will be more favourable against a discretized or aggregate output as it will allow more direct comparison with traditional survival analysis like multivariate Cox regression \citep{lin1994cox}.
\subsubsection{\label{sec}Secondary Aim: HMM}
The secondary aim would be to use either the multivariate time series of observed biosignals or their latent vector representation $o_{\{1,\cdots,T\}}$ as observed states in a Hidden Markov Model $\lambda = (A,B,\pi)$ \footnote{$A$ and $B$ are respectively the state transition matrix and the observation (emission) matrix. $\pi$ denotes initial probabilities.}. Algorithms for likelihood estimation (Forward Algorithm), decoding (Viterbi Algorithm) and learning (Baum-Welch) will be used to derive ideal parameters for the model and then extract most probable sequence of hidden states given the observations \citep{keselj2009speech}. The hidden states here can correspond to either a binary set $S = \{1 , 2\}$ corresponding to \texttt{alive} and \texttt{deceased} with their individual probabilities at time $t$ allowing us to understand the onset of HF and how it may change over a sequence of observed events.


\subsection{\label{expla}Explainability \& Interpretability}
In the medical domain, model interpretability is crucial to ensure that the predictions made by the model can be trusted and applied in a clinical setting. Although ML-based models may provide higher accuracy, their black-box nature can make them less interpretable, casting reasonable doubt on their applicability. Therefore, it is important to assess the trade-off between model performance and model interpretability. One approach to assessing this trade-off is to use model-agnostic methods such as Local Interpretable Model-Agnostic Explanations (LIME) and SHapley Additive exPlanations (SHAP), which provide insights into how the model arrived at its prediction. I plan to use these methods to explain the predictions made by the models and compare the level of interpretability with traditional survival analysis models. I can also assess the performance of the models against the level of interpretability, enabling me to evaluate the trade-off between model performance and interpretability. The results will provide guidance on the most suitable model for predicting outcomes in frail heart failure patients in a clinical setting.

\section{\label{gov}Data Governance \& Ethics}
Ethics approval is an essential requirement for this research project as it involves patient data. Patient data is considered sensitive information and approval ensures that this data is handled with care and confidentiality. For the purpose of this project, the medical school at the University of Exeter has consented to share CPRD Aurum data that they have procurred for their own research. Ethical approval is being sought from University of Exeter's management system, WorkTribe. Access to the contingent data source, MIMIC-III is restricted and requires a formal application process to maintain data privacy and integrity. To comply with this, I have undergone the CITI training program for Data or Specimens Only Research.

\section{\label{risk}Risk Assessment}
There are a few potential risks to consider before embarking on this project. Because of the sensitivity of the CPRD data, it involves some degree of navigating through bureaucratic processes of data access and retrieval. There is a chance that the data does not become available within a favourable time period. To mitigate this, access to the open-source dataset MIMIC-III has already been secured. As stated earlier, this work deals with patient data (albeit anonymized) and hence, ethics approval is necessary. As of writing this proposal however, this is still pending due to unforeseen technical glitches on the University website that is barring me from getting access to the platform. However, talks are ongoing with the University IT department so that this may be resolved at the earliest.

\section{\label{plan}Project Plan}
The project plan is outlined by the following chart.

\begin{turn}{-90}
\begin{ganttchart}[
   vgrid={*{3}{gray, dotted}, *{1}{black, dashed}},
   bar label node/.append style={
     align=left,
     text width=width("Exploratory Data Analysis")}
   ]{1}{14}
\gantttitle{May}{4} \gantttitle{Jun}{4} \gantttitle{Jul}{4} \gantttitle{Aug}{2} \\
\ganttbar{1. Literature Review}{1}{5} \\
\ganttbar{2. Exploratory Data Analysis}{2}{5} \\
\ganttbar{3. Model - Build base Layers}{4}{6} \\
\ganttlinkedbar{4. Model - Build top layer}{6}{9}\ganttnewline
\ganttlinkedbar{5. Model - Testing}{7}{11} \ganttnewline
\ganttbar{6. Model - Set up baselines}{5}{7} \\
\ganttbar{7. Thesis Writing}{8}{12} \ganttnewline
\ganttlinkedbar{8. Submission}{13}{13}
\label{fig:gantt}
\end{ganttchart}
\end{turn}	

\section{\label{c}Conclusion}
This project will leverage CPRD's rich data source to develop a robust predictive model for hospitalization and mortality in frail heart failure patients, with the ultimate goal of assessing the trade-off between performance and interpretability arising from the two different paradigms of HDL and HMM.


%\nocite{*}

%\bibliography{apssamp}% Produces the bibliography via BibTeX.

\cite{*}

\bibliographystyle{apalike}
\bibliography{20230413}% Produces the bibliography via BibTeX.

\end{document}
%
% ****** End of file apssamp.tex ******
